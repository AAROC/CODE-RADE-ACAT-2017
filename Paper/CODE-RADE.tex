\documentclass[a4paper]{jpconf}
\usepackage{graphicx}
\usepackage{hyperref}
\usepackage{enumitem}
%\usepackage[scaled=0.85]{beramono}



\begin{document}
\title{CODE-RADE a user centric system for delivery of science software.}

\author{S H T Murray$^1$ and B Becker$^2$}

\address{$^1$ Center for High Performance Computing, CSIR, Rosebank, Cape Town, South Africa}
\address{$^2$ Meraka Cyberinfrastructure, CSIR, Lynwood, Pretoria, , South Africa}

\ead{$^1$smurray@csir.co.za}
\ead{$^2$bbecker@csir.co.za}

\begin{abstract}
Scientific computing can be distilled to the execution of sets 
of applications that are combined into complex workflows. Due to both the complexity and number of 
scientific packages and computing platforms, delivering these applications to end users has always 
been a significant challenge through the grid era, and remains so in the cloud era. In this contribution 
we describe a platform for user-driven, continuous integration and delivery of research applications 
in a distributed environment - project CODE-RADE. Starting with 6 hypotheses describing the problem at 
hand, we put forward technical and social solutions to these. Combining widely-used and thoroughly-tested 
tools, we show how it is possible to manage the dependencies and configurations of a wide range of 
scientific applications, in an almost fully-automated way. 
% The CODE-RADE platform is a means for 
%developing trust between public computing and data infrastructures on the one hand and various 
%developer and scientific communities on the other hand. Predefined integration tests are specified 
%for any new application, allowing the system to be user-driven. This greatly accelerates time-to-production 
%for scientific applications, while reducing the workload for administrators of HPC, grid and cloud 
%installations. 
Finally, we will give some insight into how this platform could be extended to address 
issues of reproducibility and collaboration in scientific research in Africa.
\end{abstract}

\section{Introduction}
Since the first computing program exceeded the capabilities of its original execution host, and was 
attempted to be migrated to another platform whether it be bigger or in another building, 
scientists have suffered trying to get their research done. Countless hours have been wasted 
getting programs to run as opposed to actually doing the science they intended to do.

    It is a long standing problem and irritation trying to get a program to run on
    computing infrastructure that is not sitting on your desk. The scale and complexity being 
    propotional to that of the code and the computing system you wish to run on.

	\subsection{The paradox of  plentiful computing}

    It is a central theme of scientific computing that there are never enough resources available to
    any particular researcher. Congruent to this, there has
    been massive and near-pervasive investment in HPC and distributed computing resources. There has
    been a coordinated national effort in South Africa to build a well-connected distributed computing
    platform\cite{SAGrid}, fully integrated and connected to the national research
    network\cite{SANREN}. This effort has even been extended at a trans-continental scale, via the
    signature of a memorandum of understanding between African and European infrastructure
    providers\cite{AAROC}. With the creation of the Africa-Arabia Regional Operations
    Centre\footnote{See \url{"http://www.africa-grid.org"} for details}, several sites across Africa
    are available to researchers to execute their scientific workflows, and support several
    different virtual organisations, or communities of practice. Despite this effort in
    interoperability and integration, many of these resources are under-utilised at any given time.
    This seems to point to a paradox of scientific computing, where there is an over-provision of
    resources, yet scientists are still impeded from executing their calculations.

	\subsection{Barriers to Entry}

    This situation exists due to various barriers to entry, one of which is the ability
    to actually execute the code. One of the main benefits of a distributed computing
    platform is also one of the main barriers to entry : the complexity of the environment.
    Researchers seldom need to take into account the configuration of their application or workflow
    for different platforms, and it is practically unfeasible to insist on homogeneous hardware and
    configurations across the sites. This results in significantly increased overhead on the part of
    the researcher to ensure that their application will properly execute at any of the given
    resources, which thus leads to oversubscription at certain sites and underutilisation at
    others\footnote{Another effect is the reduction in the capacity of the researcher and experts at
    sites to effectively collaborate with each other, due to geographical and technical
    divergence.}.

    CODE-RADE aims to reduce or remove the barrier to entry for new applications, thus eliminating
    the paradox of plentiful computing.

	\section{CODE-RADE Design and Philosophy}

    The porting and integration of new applications is ordinarily done by site administrators who
    attempt to compile and install the researcher's applications according to the local procedures and
    policies. This usually leads to a significant lead-time and depends critically on the
    ability of the site administrator to interpret the needs of the researcher. There
    can also be much ambiguity regarding the resolution of dependencies, optimisation and versions. We
    then come up against a tension between domain expertise and administrative privileges. The 
    {\it researcher}, not the site administrator is usually the one that best understands how to compile the
    application. The site administrator has the knowledge of the local setup
    which would integrate the applicaiton appropriately and efficiently. 
    This overhead is multiplied by the number of distinct application stacks which
    are in use in scientific domains, and exacerbated by the multitude of dependencies and
    optimisation configurations. For the physical sciences domain, this can be appreciated by
    considering the wide array of applications registered in the EGI Applications
    Database\cite{EGIAppDB}. \footnote{For physics alone,
    there is a complexity and scale which is prohibitive to any particular site administrator.
    Considers the applications maintained in the Homebrew or EasyBuild repositories -
    $\tilde 600$ and $\tilde 1000$ respectively - the situation seems untenebale.}

%	\begin{figure}[h]
%		\begin{center}
%			\includegraphics[width=0.75\textwidth]{physical-sciences-applications-egi-appdb.eps}
%            \caption{\label{PhysicalSciencesEGIAppDBI}Graph showing the number and distribution of
%            applications in the Physical Sciences domain registered in the EGI Application
%            Database.}
%		\end{center}
%	\end{figure}

	\subsection{CODE-RADE Design Hypotheses}
    CODE-RADE provides a means to transparently integrate a new application into an arbitrary site
    configuration, that is both user-driven and satisfies the constraints imposed by site
    administrators. We reconsidered the way applications are built and deployed in scientific
    research and computational infrastructure.
    This was done as far as possible from first principles, by constructing seven hypotheses of
    bringing applications to distributed infrastructure. In what follows, we discuss the
    considerations  of these design hypotheses and how they have influenced the development and
    results of the project.

	\begin{description}
		\item[Hypothesis 1: Every researcher is an application] \hfill \\
        The core aspect of initial communication between a research infrastructure provider and a
            researcher is the {\it application}. %Every effort is be made to resolve and
           % understand the needs of the researcher and engage with them on
           % technical aspects of the application.
		\item[Hypothesis 2: No software is an island] \hfill \\
        Every scientific application has a dependency list. No definition of an application can be
            complete without an explicit expression of the list of its dependencies, which should
            extend as far as possible down userspace.
		\item[Hypothesis 3: All applications need an environment] \hfill \\
        As with any organisms, biological or digital, it requires an environment in which to 
            express its functionality.%,
           % digital organisms (scientific applications) are sterile if they are not placed in an
           % environment capable of expressing them. This environment can be expressed by
           % combinations of compilers, hardware architecture, application dependencies, middleware
           % stacks, {\it etc}.
		\item[Hypothesis 4: There is more than one environment] \hfill \\
        By restricting ourselves to a single environment, we lose main advantages of distributed
            systems, and erect unncessary barriers to entry for researchers. %We should thus design a
            %system which {\it simulates} all of the available sites which applications could in
            %principle be executed, to increase application portability.
		\item[Hypothesis 5: Solutions Decay] \hfill \\
        A ``solution" in this case refers to a software expression of the method for creating an
            executable version of the application. The exploration of these solutions, including the
            attempts which failed, are important components of the system knowledge. %Due to
           % the development of software, it is likely that a found solution will not apply for
           % future iterations. Thus, solutions need to be expressed in a machine-readable form {\it
           % which is maintained in a version-controlled repository}
        \item[Hypothesis 6: Humans need not apply] \hfill \\%\footnote{With apologies to CGP Grey - https://www.youtube.com/watch?v=7Pq-S557XQU}] \hfill \\
        If we assume that the previous hypotheses are true, it stands to reason that most of the
            work to actuall compile and integrate applications can be automated. %This represents a
            %major relief of workload from site administrators, as well as ensuring that human errors
            %are less frequently introduced. Furthermore, if a system is built with respecting open
            %standards, it becomes easier to extend it.
		\item[Hypothesis 7: This is not hard] \hfill \\
        While it may seem that we are proposing a radical shift from common practice, which may seem
            discouraging, the fact is that continuous integration\cite{ContinuousIntegration} and
            continuous delivery\cite{ContinuousDelivery} has been common practice in the software
            development world for several years.
	\end{description}

    Along with these hypotheses, CODE-RADE attempts  to solve problems which impede fluid
    collaboration and exploitation of computational resources. First of all, the system aims to
    reduce or remove entire barriers to entry represented by user applications, through the use of
    continuous integration and delivery, automated builds and version-controlled source code
    repositories. This allows all involved to know exactly where the application is in its
    lifecycle, and indeed which build of the application is currently being used.

    Automated continuous delivery ensures that once applications are integrated, they are
    immediately available at sites which wish to host them, further reducing the workload on site
    administrators.

	\section{CODE-RADE in detail}

	\subsection{Actors}

	The CODE-RADE platform identifies four kinds of actors which have distinct roles to play :
	\begin{itemize}
		\item Researchers
		\item Research Software Engineers (RSE)
		\item Infrastructure Operators (Ops)
		\item Automated Agents (Bots)
	\end{itemize}

    Researchers are the final users of the application, they are not necessarily the authors of
    the software. Researchers may not know the details necessary to compile and optimise
    applications for specific environments, they are adept in interpreting the results of
    these applications and are the end consumers of them. The research software engineer, which in
    some cases is the researcher themselves, thus provides a means to express the application on the
    various targets in an executable state. Proof positive to the resource providers that
    applicatons can indeed execute properly on target infrastructures is needed before they will
    agree to actually host and excute the applications on their sites. Indeed, the characteristics
    of these target infrastructures are defined by the infrastructure operations team, which know
    them best. The role of the automtaed agents is to execute the code provided by the research
    software engineers and see whether it passes the tests set by the infrastructure operations
    team, at every commit pushed to the repository. This provides a direct link between change in
    the software and expression of that software (source code version linked to executable version),
    and since the tests criteria are predefined, once they pass applications can be integrated
    without human intervention.

	In principle, the entire workflow can be driven by researchers themselves.

	\subsection{Platform}

    In this section, we describe how we have built the CODE-RADE platform, based on the philosophy,
    hypotheses, and actors described above.

	\subsubsection{Components}

	The implementation of CODE-RADE consists essentially of three generic components :

	\begin{itemize}
		\item Continuous integration tool
		\item Version controlled software repository
		\item Automated software delivery tool
	\end{itemize}

    There are several choices  which one could use to build a similar platform to CODE-RADE by
    choosing a particular tool to fulfill each component. In our case, we have selected
    Jenkins\cite{Jenkins}, GitHub\cite{Github} and CERN Virtual Machine Filesystem\cite{CVMFS}. The
    reasoning and justification for these specific choices is given in some more detail below in
    section \ref{Discussion}. The components have specific roles to play in each phase of the
    CODE-RADE workflow, which we describe next.

	\subsubsection{CODE-RADE workflow}

    Although the main aim of CODE-RADE is to ensure that applicaitons are available for execution on
    arbitrary remote sites, there are collateral issues which have importance to a community. We
    identify several goals in CODE-RADE which have their own flow of actions and we thus define the
    generic workflow for application {\bf integration, delivery, review} and {\bf publication}.\\

	The most common workflow is 3 steps : Build, Test, Deliver.  
    The first step (build) this refers to the synchronisation of the source code with relevant
    updates. This is initiated by the Research Software Engineer: a commit is made to a repository,
    detailing a change which the RSE authors. This triggers an event on Jenkins, which pulls the
    latest commit and executes the scripts in the repository which will build and test the code. The
    second step (Test) is conditionally executed on success of the previous step. This step checks
    the viability of the built application and executes whatever tests are specified, then installs
    the application into the integration environment. Finally the third step (Deliver) cleans the
    built application and installs it into the deploy environment, then ships the executable code to
    the repository for subsequent replication.

    Although there are several tools for building widely-used applications (see Section
    \ref{Discussion} below), we have chosen a convention within the project to have three scripts
    which perform the compilation, functional testing and integration into the delivery environment
    respectively. 
    
% TODO - Add figure of the workflow

Each job, as defined on Jenkins, consists of at least these three tasks. If each passes successfully,
a job to update the CVMFS repository is triggered, which puts the repository into transaction, pulls
in the newly built applications and module files, then publishes a new version of the repository.
This is automatically and transparently available at sites after a few seconds, once the CVMFS
client detects the change in the repository hash. In order to make it easier for end users or client
applications to check which version of the repository they are using\footnote{This can be determined
more accurately using the cvmfs client tools, but this may not be intuitive to users.}, an integer
increment is added to a file in the repository which is linked to the job which resulted in the
latest version.

In short, a commit to a repository by a researcher or RSE will result in the application being
automatically delivered to arbitrary endpoints, as long as quality and functional tests set by Ops
and researchers respectively are passed.

	\section{Discussion}\label{Discussion}

Our particular combination of existing tools (Github, Jenkins and CVMFS) has resulted in a powerful
platform with which to exploit any existing computational infrastructure, in an automated
user-driven way. Some aspects which together make CODE-RADE a novel and powerful platform should be
highlighted.

\subsection{CODE-RADE is cross-platform}

CODE-RADE takes a software description of an scientific tool and builds a digital expression which
can be executed on arbitrary architectures. This makes CODE-RADE different from more traditional
package managers, since these only build for the architecture on which the user is currently. By
simulating the execution environment which may in principle be available at geographically
distributed sites. We currently define a target according to a matrix of characteristics, which are
encoded as variables in the build and deploy environments :

\begin{description}
	\item[ARCH]: The CPU architecture of the execution environment
	\item[OS]: The operating system of the execution environment
    \item[SITE]: A meta-variable encompassing aspects of specialised hardware (GPU, Myrinet, GPFS,
        etc) and other aspects (software licenses, available middleware etc.)
\end{description}

For each combination of these axes, specific optimisations can be applied, ensuring that the final
result (compiled application) is both properly compiled, and will execute efficiently on the target
site.
By building on as many environments as possible, for as many combinations as possible, CODE-RADE
promotes diversity in computing platforms, without exacerbating the paradox of plentiful computing.

\subsection{CODE-RADE is atomic}

Successfully built git hashes are linked to a cvmfs hash, permitting reproducibility across sites and 
allowing one to link research outputs to the underlying versions and builds. This is then citable via
Zenodo digital object identifiers (doi) from the cvmfs hash, permitting the work done on the 
infrastructure to be citable and credit given, and critically be reproducible.
The researcher has fine-grained control over dependencies, version, and targets that are built
and built against. 

\subsection{CODE-RADE is community-based}
There is no restriction on the applications that can be integrated, so long as the software is accesable
online in someway. Anyone can contribute applications, resources, code review, testing, or
where ever their skill sets lie. 
The entire community can see the build status and reinitiate failed builds via new pull requests.
After a user updates their git repository, a git pull request is done, a build is executed. 
The results of the build are publicly viewable, 
errors are viewable by the user doing the pull request (and everybody else), via a full build transcript.
This can be fixed
by merely redoing a pull request with what ever changes are required to fix the errors. 
This is completely indepdent of staff at sites.

\subsection{CODE-RADE is Automated}

There is a heavy relilance on automation in CODE-RADE, via automated agents to reduce bias, lead time
and is user-driven.
As an example, failures on jenkins cause issues on github, and the user is then able to fix these at
their discretion.
Successful subsequent builds will also close these issues.
Notifications are sent to relevant slack channels (slack.com) and bots within slack permit actions
to be taken, for instance postponing issues, reassigning, or closing.

\section{Summary, Conclusion and Futher work}

Making the best use of computational infrastructure comes down to running applications.
Maintaining and porting them is a hard and often thankless task.
We think we have come up with a solution to solve many parts of this and making the work citable.
We've built an automated porting system, which will deliver functional, tested, relevant software to
your site that is reproducible. The work done is then citable to give credit where credit is due for work
that often goes uncredited.
It eases the communication blocks in collaborative work, often highly distributed.
Everyone is welcome to come on in and help us build it.

\section*{Acknowledgements}
The authors would like to thank contributors to the project for ideas, discussion and code
The original idea co-developed by one of the authors (BB) and Fanie Riekert (University of the Free
State). Input and critique was provided  by Dane Kennedy and Sakhile Masoka (Centre for High
Performance Computing, Cape Town) and Peter van Heusden (South African National Bioinformatics
Institute). The authors recognise the technical support of the EGI CVMFS Task Force support, in
particular Catalin Condurache (European Grid Initiative and Science and Technology Facilities
Council, UK). Several discussions on CODE-RADE design and extension were had with Timothy Carr
(University of Cape Town e-Research Centre).

CODE-RADE is supported by the Sci-GaIA project under grant 654237 of the European Commission's
Horizon 2020 programme

	\section*{References}
	\begin{thebibliography}{9}
		\bibitem{SAGrid}  ``The South African National Compute Grid" B. Becker {\it et al.} IST-Africa 2009 Conference Proceedings, Paul Cunningham and Miriam Cunningham (Eds), IIMC International Information Management Corporation, 2009, ISBN: 978-1-905824-11-3
		\bibitem{SANREN} ``SANReN – South African National Research Network." [Online]. Available: http://www.sanren.ac.za/. [Accessed: 25-Jul-2016].
		\bibitem{SANREN-thesis} ``The conceptualisation, design and implementation of a national research and education network", S. A. Brink Master's thesis, University of Stellenbosch. http://hdl.handle.net/10019.1/1516
		\bibitem{AAROC} ``Mou between EGI.eu and CSIR Meraka (African-Arabian Region)", Sergio Andreozzi. Mou between EGI.eu and CSIR Meraka (African-Arabian Region), 15-Apr-2015. [Online]. Available: \url{https://documents.egi.eu/public/ShowDocument?docid=2407}. [Accessed: 25-Jul-2016].
		\bibitem{EGIAppDB} ``EGI Applications Database", EGI Applications Database. [Online]. Available: https://appdb.egi.eu. [Accessed: 25-Jul-2016].
		\bibitem{ContinuousIntegration} Paul M. Duvall, ``Continuous Integration". 2007. Pearson Educational Publishers.  ISBN-10: 9780321336385
		\bibitem{ContinuousDelivery} Humble, Jez and Farley, David, ``Continuous delivery: reliable software releases through build, test, and deployment automation". 2010. ISBN 10: 0-321-60191-2
		\bibitem{Jenkins} J. F. Smart, ``Jenkins: The Definitive Guide". O’Reilly Media, Inc., 2011. ISBN 1-4493-0535-0 978-1-4493-0535-2
		\bibitem{Github} See https://developer.github.com/v3
		\bibitem{CVMFS} J. Blomer, P. Buncic, and T. Fuhrmann, ``CernVM-FS: Delivering Scientific Software to Globally Distributed Computing Resources," in Proceedings of the First International Workshop on Network-aware Data Management, New York, NY, USA, 2011, pp. 49–56.
    \bibitem{CMake} Ken Martin and Bill Hoffman, ``An Open Source Approach to Developing Software in a Small Organization". In IEEE Software, Vol. 24 Number 1 IEEE, January 2007.
    \bibitem{AutoTools} ``GNU Automake - GNU Project - Free Software Foundation (FSF)." [Online]. Available: https://www.gnu.org/software/automake/manual/. [Accessed: 28-Jul-2016].


	\end{thebibliography}
	% \bibliography{Zotero2}{}
	% \bibliographystyle{plain}
\end{document}



