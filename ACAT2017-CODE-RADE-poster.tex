%%
%% This is file `tikzposter-example.tex',
%% generated with the docstrip utility.
%%
%% The original source files were:
%%
%% tikzposter.dtx  (with options: `tikzposter-example.tex')
%% 
%% This is a generated file.
%% 
%% Copyright (C) 2014 by Pascal Richter, Elena Botoeva, Richard Barnard, and Dirk Surmann
%% 
%% This file may be distributed and/or modified under the
%% conditions of the LaTeX Project Public License, either
%% version 2.0 of this license or (at your option) any later
%% version. The latest version of this license is in:
%% 
%% http://www.latex-project.org/lppl.txt
%% 
%% and version 2.0 or later is part of all distributions of
%% LaTeX version 2013/12/01 or later.
%% 




 \documentclass[25pt, a0paper, portrait, margin=0mm, innermargin=15mm,
     blockverticalspace=15mm, colspace=15mm, subcolspace=8mm]{tikzposter} %Default values for poster format options.

 \tikzposterlatexaffectionproofon %shows small comment on how the poster was made at bottom of poster

 % Commands
 \newcommand{\bs}{\textbackslash}   % backslash
 \newcommand{\cmd}[1]{{\bf \color{red}#1}}   % highlights command

 % Title, Author, Institute
 \title{CODE-RADE}
 \author{Sean Murray, \& Bruce Becker}
 \institute{Council for Scientific and Industrial Research, Meraka. University of Cape Town}
\titlegraphic{img/coderade.png}

 % -- PREDEFINED THEMES ---------------------- %
 % Choose LAYOUT:  Default, Basic, Rays, Simple, Envelope, Wave, Board, Autumn, Desert,
\usetheme{Simple} % Board, Simple, Wave, Rays.
\usecolorstyle[colorPalette=BrownBlueOrange]{Germany}


 \begin{document}

     \maketitle{}
% basic structure
% Summary of arch and tech
% schematic of code-rade workflow, the actors etc.
% diagram of delivery 
     % map of whereit can be run (grid sites)
% description of repos and repo layouts.
% todo and next features

     \begin{columns}%blocks will be placed into columns
         \column{.5}
         \block[roundedcorners=40]{7 Hypotheses of scientific computing}{
             \begin{itemize}
                 \item It always comes down to an application
                 \item No software is an island
                 \item Applications require environments.
                 \item There are multiple environments
                 \item Solutions decay 
                 \item Remove humans
                 \item Stuff is not hard, leverage what is around

             \end{itemize}

         }

         \block[roundedcorners=40]{Achitecture }{
             \begin{itemize}
                 \item I
                 \item S
             \end{itemize}

         }
         \column{0.5}
         \block[roundedcorners=40]{Technology  }{
             \begin{itemize}
                 \item I
                 \item N
                 \item A
                 \item S
                 \item R
                 \item S
             \end{itemize}
        }
     \end{columns}

     \block[bodyoffsety=-1cm]{Sample document}{\vspace{2em}
     %\block[titleoffsety=-1cm,bodyoffsety=-1cm]{Sample document}{\vspace{2em}
         This poster was created by the following commands (omitting the contents of the blocks and notes) to give a sense of how different objects are created and options used.
            put the flow chart in the middlehere, possibly two, with a before and an after.
     }
     % map of whereit can be run (grid sites)
% description of repos and repo layouts.
% todo and next features

     \begin{columns}%blocks will be placed into columns
         \column{.5}
         \block[roundedcorners=40]{7 Hypotheses of scientific computing}{
             \begin{itemize}
                 \item It always comes down to an application
                 \item No software is an island
                 \item Applications require environments.
                 \item There are multiple environments
                 \item Solutions decay 
                 \item Remove humans
                 \item Stuff is not hard, leverage what is around

             \end{itemize}

         }

         \column{0.5}
         \block[roundedcorners=40]{Technology  }{
             \begin{itemize}
                 \item I
                 \item N
                 \item A
                 \item T
                 \item S
                 \item R
             \end{itemize}
        }
     \end{columns}

 \end{document}




\endinput
%%
%% End of file `tikzposter-example.tex'.
